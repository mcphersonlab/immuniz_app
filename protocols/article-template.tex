%%%%%%%%%%%%%%%%%%%%%%%%%%%%%%%%%%%%%%%%%%%%%%%%%%%%%%%%

%%%  Welcome to the Cell Press STAR Protocols LaTeX template,     
%%%  version 1.0. This is a minimalist template    
%%%  to help you organize your article for            
%%%  publication at Cell Press. PLEASE NOTE:

%%%  (1) If you submit your final manuscript materials 
%%%  in LaTeX format, our typesetters will prepare 
%%%  a Word file for use in production. This conversion 
%%%  process allows us to copyedit your paper, fix 
%%%  any typos, and add formatting and tagging. The 
%%%  conversion process will add approximately 3 
%%%  business days to the production timeline. 
%%%  Authors using LaTeX should keep this in mind 
%%%  when considering deadlines.

%%%  (2) Keep your LaTeX files as simple as possible. 
%%%  Avoid the use of elaborate local macros and/or 
%%%  customized external style files. If you need 
%%%  additional macros, please keep them simple and 
%%%  include them in the actual .tex document preamble. 
%%%  Source code should be set up so that all .sty 
%%%  and .bst files called by the main .tex file are 
%%%  in the same directory as the main .tex file. 

%%%  (3) Please review the formatting guidelines/IFA: 
%%%  https://www.cell.com/star-protocols/authors.

%%%  Please send feedback on this template to lshipp@cell.com. 

%%%%%%%%%%%%%%%%%%%%%%%%%%%%%%%%%%%%%%%%%%%%%%%%%%%%%%%%

\documentclass[12pt,letterpaper]{article}
\usepackage[a4paper, total={7in, 10in}]{geometry}
\renewcommand{\familydefault}{\sfdefault}
\usepackage{graphicx}
\usepackage{helvet}
\usepackage[dvipsnames]{xcolor}
\usepackage{authblk}
\usepackage{hyperref}
\usepackage{amsmath} 
\usepackage{amssymb} 
\usepackage{orcidlink} 
\usepackage[super,comma,sort&compress]  
   {natbib}\bibliographystyle{numbered}
\usepackage[right]{lineno} \linenumbers


\makeatletter
\renewcommand{\maketitle}{\bgroup\setlength{\parindent}{0pt}
\begin{flushleft}
  \textbf{\@title}
  
  \@author
\end{flushleft}\egroup}
\makeatother

%%%  Insert title below; leave date empty. Choose a concise title that includes the name of the 
%%%  method/technique, the purpose of the protocol, and the model system or organism. 

\title{Title is a maximum of 145 characters, including spaces, and avoids the use of jargon and uncommon abbreviations}
\date{}

%%%  Author first and last names should be spelled 
%%%  out in their entirety (do not abbreviate "J.H. 
%%%  Watson" unless this is how the author's name 
%%%  always appears). Middle initials are OK. Do 
%%%  not include titles, positions, or degrees.

%%%  Use numbered footnotes to indicate institutional 
%%%  affiliations. Authors may have multiple 
%%%  institutional affiliations, and affiliations 
%%%  may be shared among multiple authors.

%%%  After the institutional affiliations, numbered 
%%%  footnotes may be used to indicate an author's 
%%%  present address, equal contribution status, 
%%%  and/or senior author status. Corresponding 
%%%  authors may additionally include Twitter (X) 
%%%  handles as a means of contact.

%%%  The last numbered footnotes should indicate 
%%%  which author is the technical contact and which author 
%%%  is the lead contact. Both are required. 
%%%  One author must be designated as the lead contact. 
%%%  There can be no more than one lead contact. 

%%% Explanations of the roles and responsibilities of the corresponding author(s) and lead and    
%%% technical contacts can be found in the authorship section of the information for authors. Please 
%%% include email addresses for both the lead and technical contact in the Resource availability  
%%% section. Please note that: The technical contact must match the name/contact in the Resource     
%%% availability statement.


%%%  Corresponding authors should be indicated with 
%%%  asterisks (*). Use 2 asterisks (**) for the 
%%%  second-listed corresponding author, 3 (***) for 
%%%  the third-listed, and so on. The lead contact 
%%%  must be a corresponding author. Additional 
%%%  authors may also serve as corresponding authors.

\author[1,5,\orcidlink{0000-0000-0000-0000}]{First Last}
\author[1,2,5]{First Middle Last}
\author[2,3]{First M.M. Last}
\author[3,*]{First Last, Jr.}
\author[3,4,6,7,8,**]{F. Middle Last}


%%%  Institutional affiliations should contain the 
%%%  following information at minimum: department(s)/
%%%  subunit(s), institution, city, state/region (if 
%%%  applicable), and country. 

\affil[1]{University A, London SW7 2AZ, UK}
\affil[2]{Department B, University B, Toronto, ON M5S 3H6, Canada}
\affil[3]{Division C, Department C, University C, Cambridge, MA, USA}

\affil[4]{Present address: 123 Sesame Street, New York, NY, USA}
\affil[5]{These authors contributed equally}
\affil[6]{Senior author}
\affil[7]{Lead contact}
\affil[8]{Technical contact}

%%%  List only one email address per corresponding author.

\affil[*]{Correspondence: first.last.jr@university.edu}
\affil[**]{Correspondence: f.middle.last@university.edu}


\begin{document}

\maketitle

\section*{\color{NavyBlue} SUMMARY}
\begin{itemize}
\item The summary is a paragraph no longer than 80 words and written in the active voice and present tense. 
\item Any background information should be limited to one sentence. The summary should focus on the details of the major steps of the protocol, techniques involved, and model organisms used.
\item Avoid the use of the word “method” in your summary.  Acceptable substitutions are “protocol”, “technique”, “assay” or “approach”. Do NOT include references in the summary.
\item Summaries should include the statement as the last sentence: For complete details on the use and execution of this protocol, please refer to X et al.\textsuperscript{1}
    \begin{itemize}
        \item Only reference published, peer-reviewed and associated manuscripts in this statement.  If you do not have an associated published manuscript, do not include this statement, and make note in your submission cover letter.
        \item This statement does not count towards the 80-word limit for the Summary.
    \end{itemize}


\end{itemize}
\section*{\color{NavyBlue} Graphical abstract}
\begin{itemize}
    \item The GA is a graphical description of the protocol that highlights the time for each major step. Click \href{https://www.cell.com/pb-assets/journals/EM/STAR Protocols/STARProtocolsExampleGraphicalAbstracts.pdf}{here} to see examples. We strongly encourage you to submit a GA with your initial submission, but it is not required; however, it will be required if your paper is accepted.
    \item We have created a \href{https://www.cell.com/pb-assets/journals/EM/STAR%20Protocols/STAR_GA_Template_SP_Finalized-1694695050470.pptx}{graphical abstract template} (**if the link is not working, please copy and paste \href{https://www.cell.com/pb-assets/journals/EM/STAR Protocols/STAR_GA_Template_SP_Finalized-1694695050470.pptx}{this link} to your web browser to help you create a GA. 
    \item Programs such as \href{https://www.biorender.com/}{Biorender} may also be used with licensing and citation to create the Graphical Abstract.
    \item Please ensure that the GA is in jpg format.
\end{itemize}
\section*{\color{NavyBlue} Before you begin}
\begin{itemize}
    \item Describe the protocol for a specific usage case (ideally what was done in your associated research paper).  You may also describe alternate applications of the protocol.
        \begin{itemize}
            \item For example: “The protocol below describes the specific steps for using HeLa cells. However, we have also used this protocol in HEK293 cells and NIH3T3 cells.”
        \end{itemize}
    \item Keep background information related to the protocol to a minimum. Introductory text should not exceed 1 page.
    \item Provide instructional steps for what needs to be set up or prepared before a researcher begins the protocol. Please \textbf{use numbering} (up to three levels: 1,2,3; a,b,c; i,ii,iii) to list steps. \textbf{Do not use bullet points} in this section.
        \begin{itemize}
            \item Avoid including more than 15 sub-steps for an individual step.
        \end{itemize}
    \item \textbf{All materials recipes should be listed in the Materials and Equipment Setup section.  Do not place them here. }  
    \item For explanatory text related to the steps, please use the following callouts that can be inserted after the relevant numbered step.
        \begin{itemize}
            \item Call outs are not to contain instructional steps.
            \item \textbf{Optional}: [Call out optional steps in this format; optional steps are not numbered.]
            \item \textbf{Note}: [Please include general notes or explanations as a note. Notes are not numbered and should be near the relevant step.]
            \item \textbf{Pause point}: [If relevant, list potential pause points with time intervals in this format; pause points are not numbered.]
            \item \textbf{CRITICAL}: [when applicable, note steps that are critical for success of the protocol; critical steps are usually not numbered.]
            \end{itemize}
\end{itemize}

\section*{\color{NavyBlue} Institutional permissions (if applicable)}
\begin{itemize}
    \item Any experiments on live vertebrates or higher invertebrates must be performed in accordance with relevant institutional and national guidelines and regulations. 
    \item In the before you begin section, a statement identifying the committee approving the experiments and confirming that all experiments conform to the relevant regulatory standards must be included.
        \begin{itemize}
            \item In addition, you should provide a statement reminding readers that they will need to acquire permissions from the relevant institutions.
        \end{itemize}
\end{itemize}

\section*{\color{NavyBlue} Preparation one}
\textbf{Timing: s, min, h, days, or weeks.}
\begin{enumerate}
    \item First step of preparation one. Use numbers to order steps.
    \item Second step of preparation one.
        \begin{enumerate}
            \item Sub-steps under first step. Use letters for ordering of sub-steps.
            \item Sub-step two, as needed.
                \begin{enumerate}
                    \item Third-level sub-steps under second-level sub-steps. Use lowercase Roman numerals for ordering.
                    \item Third-level sub-step two, as needed.
                \end{enumerate}
        \end{enumerate}
\end{enumerate}
\textbf{CRITICAL:} when applicable, note steps that are critical for success of the protocol.

\section*{\color{NavyBlue} Preparation two}
\begin{enumerate}
    \item First step of preparation two. Use numbers to order steps.
        \begin{enumerate}
            \item Use letters for ordering of sub-steps.
            \item Sub-step two, as needed.
        \end{enumerate}
    \item Second step of preparation two.
\end{enumerate}

\section*{\color{NavyBlue} Key resources table}
%%% The Key Resources Table (KRT) highlights key reagents, resources, and equipment used in the 
%%% protocol. Include vendor/manufacturer and the model/catalog number for each reagent or resource. 
%%% Whenever possible, use RRIDs as the identifiers for the items used in the protocol (see 
%%% https://scicrunch.org/resources). •	DO NOT CHANGE THE HEADINGS IN THE KRT. For 
%%% reagents/resources that are critical for the success of this protocol, please provide links to 
%%% the corresponding vendor sites. Please note that the heading "Other" is for items not covered by 
%%% other headings and all equipment (microscopes, flow cytometers, etc). Please do not leave blank 
%%% spaces on the key resources table (use n/a if there is no information to include). Also, note 
%%% that only the standardized subheadings can be used within the table. The KRT can be inserted 
%%% into the manuscript (preferred method), or alternatively uploaded as a separate document.



\textit{To create the KRT using word, please use the \href{https://star-methods.com}{KRT webform} or this \href{https://www.cell.com/pb-assets/journals/research/cell/methods/table-template1-1699013648137.docx}{table template}.}

\section*{\color{NavyBlue} Materials and equipment setup (optional)}
\begin{itemize}
    \item This section is for providing additional information regarding equipment setup, details about custom software used, and notes on alternative materials and equipment. This optional section is a complement to the key resources table. 
    \item ALL EQUIPMENT SHOULD BE LISTED IN THE KRT UNDER “OTHER,” NOT HERE.
    \item All recipes should be in this section; these tables do not require legends or call outs and can stay in this section. 
    \item Please use bullet points (up to three level) to list items. Do not use numbering (1,2,3; a,b,c; i,ii,iii) in this section.
    \item Recipes with 3 or more ingredients should be presented as tables. All solutions should include storage conditions (temperature and maximum store time).
\end{itemize}

\section*{\color{NavyBlue} Step-by-step method details}
\begin{itemize}
    \item This section lists the major steps and provides step-by-step details and timing for each major step. 
    \item Please use continuous numbers for this section through Major Steps.  Do not continue numbering from the Before You Begin section.
    \item Please use numbering (up to three levels: 1,2,3; a,b,c; i,ii,iii) to list steps. Do not use bullet points in this section.
        \begin{itemize}
            \item Avoid including more than 15 sub-steps for an individual step.
        \end{itemize}
    \item DO NOT number the Major Steps.  
    \item When describing the protocol steps, write in the present tense and use active and imperative verbiage throughout.  See authors instructions page for an example.
    \item For steps using manufacturer's protocols, please include link(s) to the protocols.
    \item Whenever possible, include links between troubleshooting sections and major steps.
\end{itemize}

\section*{\color{NavyBlue} Your major step one}
%%% Your major step title should be descriptive. Avoid short titles such as “Cell culture”, 
%%% “Western blot”, or “qPCR”. Include a brief description (no more than 2 sentences) about 
%%% what this major step accomplishes; this will help other researchers repeat and troubleshoot 
%%% the protocol. Timing must be included with each major step. It is also possible to include 
%%% additional timing notes for sub steps, but only if the total time for the major step is 
%%% included.  For time ranges that are difficult to define, you may use ranges or “variable”.

\begin{enumerate}
    \item First step of major step one. Use numbers to order steps
        \begin{enumerate}
            \item Sub-steps under first step. Use letters for ordering of sub-steps.
            \item Sub-step two, as needed.
                \begin{enumerate}
                    \item Third-level sub-steps under second-level sub-steps. Use Roman numerals for ordering.
                    \item Third-level sub-step two, as needed.
                \end{enumerate}
        \end{enumerate}
    \item Second step of major step one.    
\end{enumerate}

\section*{\color{NavyBlue} Your major step two}
Include a brief description about what this major step accomplishes. This will help other researchers repeat and troubleshoot the protocol.
\begin{enumerate}
    \item First step of major step two. Use numbers to order steps.
        \begin{enumerate}
            \item Sub-steps under first step. Use letters for ordering of sub-steps.
            \item Sub-step two, as needed.
        \end{enumerate}
    \item Second step of major step two.    
\end{enumerate}

\section*{\color{NavyBlue} Expected outcomes}
This section should be written in paragraph form and detailed the expected results from the protocol and what a researcher can expect to produce for data. Include information about anticipated outcomes of the protocol, e.g., estimated yield of DNA extraction, images of protein expression pattern. We encourage authors to provide figures to illustrate the expected outcomes (see examples \href{https://www.cell.com/pb-assets/journals/EM/STAR%20Protocols/STARProtocolsExampleFigures.pdf}{here}) and describe the expected results in paragraphs.If you have no associated primary research manuscript, you are REQUIRED to provide evidence to validate the data generated by your protocol. DO NOT directly insert relevant results sections OR directly reproduce figures from a primary research paper. Find additional instructions for figure preparation in our instructions for authors \href{https://www.cell.com/star-protocols/authors}{here}.

\section*{\color{NavyBlue} Quantification and statistical analysis (optional)}
This section should be written in paragraph form and detail the analysis pipeline. Avoid placing extensive stepwise instructions in this section; if the analysis methods require multiple stepwise instructions, add them as a detailed analysis major step to the protocol. Either numbering (up to three levels: 1,2,3; a,b,c; i,ii,iii) or bullet points (up to three levels) can be used to list steps in this section. Provide details about the methods used for data processing, quantification, and statistical analysis of the data generated in this protocol. Include criteria for data inclusion/exclusion. If the protocol outcome requires a complex statistical or computational analysis, include a sample data set to allow others to repeat your approach, and provide instruction on the interpretation of the raw data. You may provide sample data or reference supplementary data files here.

\section*{\color{NavyBlue} Limitations}
In paragraph form, describe possible limitations of the protocol, including any situations where the protocol may be unreliable or unsuccessful. Describe environmental factors or mechanical limitations that might affect the validity of the results.

\section*{\color{NavyBlue} Troubleshooting}
\begin{itemize}
    \item Describe any problems that might arise from running the protocol and provide possible solutions. 
    \item Make sure to flag them in the corresponding steps of the protocol and refer to this section.
    \item If applicable, this section can also be the space to include information about alternatives, i.e., what reagents and/or equipment has some flexibility and what cannot be changed. 
    \item If you wish to include a list for each problem/potential solution, please use bullet points (up to three levels) make a list. 
    \item DO NOT use numbering (1,2,3; a,b,c; i,ii,iii) in this section.
    \item We recommend including a minimum of 5 Troubleshooting points.
    \item Link all Troubleshooting points to the appropriate Steps in the manuscript.  Provide the step number in the Troubleshooting point.
\end{itemize}

\section*{\color{NavyBlue} Problem 1:}
When illustrating problems, we encourage the use of figures/videos to indicate a good outcome versus a bad outcome. See examples \href{https://www.cell.com/pb-assets/journals/EM/STAR%20Protocols/STARProtocolsExampleFigures.pdf}{here} and \href{https://star-protocols.cell.com/protocols/437}{here}.

\section*{\color{NavyBlue} Potential solution:}
Describe the details of the solution to resolve this problem.

\section*{\color{NavyBlue} Resource availability}

%%%  The lead contact should match the one on the cover page. The technical contact should match 
%%%  the one on the cover page.We require four subheadings in this section (lead contact, technical 
%%%  contact, materials availability, and data and code availability). The guidelines include 
%%%  instructions and examples.


\subsubsection*{Lead contact}


 Further information and requests for resources and reagents should be directed to and will be fulfilled by the lead contact, [lead contact’s name] (lead contact’s email).

\subsubsection*{Technical contact}

Technical questions on executing this protocol should be directed to and will be answered by the technical contact, [technical contact’s name] (technical contact’s email).

\subsubsection*{Materials availability}

Provide information regarding availability of newly generated materials associated with this protocol, including any conditions for access. 

Plasmids generated in this study have been deposited to [Addgene, name and catalog number or unique identifier].

\subsubsection*{Data and code availability}

State whether the protocol includes all datasets generated or analyzed during this study. Provide information about data and code availability.




\subsection*{\color{NavyBlue} Figures}

Do not include figures in the main text document.  All figures should be uploaded as separate high-resolution JPG. All captions for tables should also be included here.


\subsection*{\color{NavyBlue} Citations}

To cite/link to display items (figures and tables) and/or sections of the manuscript, simply write, for example, ``(Figure 1)" or ``see discussion." Our team will do the rest. Please note that we do not number sections or subsections.

To cite references, you may use the cite command, e.g., ``Recent articles in \textit{Matter} and \textit{Cell} \cite{cates1984statics,li2020conformational,wang2020conformational}  have shown ..." or  ``Many interesting discoveries have been reported, \cite{sambrook1989molecular, kapoor2022leakage,hantke2018my,gerczuk2023zenodo} which ..." 

\subsection*{\color{NavyBlue} Equations}

Simple formulae should appear in line with the text whenever possible. You can write inline math by enclosing it between \verb|\(| and \verb|\)|, as in this example: \(x^2 + y^2 = z^2\). You can also enclose it between dollar signs (\texttt{\$}), as in this example: $E=mc^2$.

Larger, more complex formulae may appear on a new line, either by enclosing them between \verb|\[| and \verb|\]| or by using the \verb|displaymath| environment:

\[ x^n + y^n = z^n \]

\begin{displaymath}
\sqrt{x^2+1}
\end{displaymath}

If any equations or formulae need to be referred to or cited again later in the text, use the \verb|equation| environment to number them. Later, you can cite these as "Equation 1," "Equation 2," etc.

\begin{equation}
f(x) = \sum_{i=0}^{n} \frac{a_i}{1+x}
\end{equation}


\newpage




\newpage

%%%  The following sections sould appear after the 
%%%  resource availability section.

\section*{\color{NavyBlue} Supplemental information index (contact your handling editor)}

%%% Each supplemental data item descriptive title should 
%%% reference a related step in the protocol. 
%%% Example: Data S1: One-sentence descriptive title for the 
%%% item, related to Step 1.


\begin{description}
  \item Figures S1-S5 and their legends in a PDF
  \item Table S1. A descriptive title for an Excel file that was too large to appear in the PDF
  \item Table S2. Another descriptive title for a different Excel file
  \item Data S1. Raw data on x, y, and z
\end{description}

\section*{\color{NavyBlue} Acknowledgments}

%%%  Enter the following information here: 1) all funding sources; 2) collaborators and/or core facilities contributing to the work.

This work was funded by [FUNDER] via grant [GRANT NO.]. The authors thank all members of the lab for their support.

\section*{\color{NavyBlue} Author contributions}

%%% Mention each individual author with a statement outlining the contribution of each author to the work.

Conceptualization, S.C.P. and S.Y.W.; methodology, A.B., S.C.P., and S.Y.W.; investigation, M.E., A.N.V., N.A.V., S.C.P., and S.Y.W.; writing – original draft, S.C.P. and S.Y.W.; writing – review \& editing, S.C.P. and S.Y.W.; funding acquisition, S.C.P. and S.Y.W.; resources, M.E.V and C.K.B.; supervision, A.B., N.L.W., and A.A.D.

\section*{\color{NavyBlue} Declaration of interests}

%%%  This section is required, even 
%%%  if the authors have no competing interests; if 
%%%  this is the case, insert "The authors declare no 
%%%  competing interests." Please refer to the 
%%%  declaration of interests policy: 
%%%  https://www.cell.com/declaration-of-interests

S.Y.W. is an employee and shareholder of COMPANY. M.E.V. is a founder of COMPANY and a member of its scientific advisory board.

\section*{\color{NavyBlue} Declaration of generative AI and AI-assisted technologies}

%%%  If generative AI and AI-assisted technologies 
%%%  were used in the writing process, this must 
%%%  be disclosed in the paper. This declaration 
%%%  does not apply to the use of basic tools for 
%%%  checking grammar, spelling, references, etc. 
%%%  If you have nothing to disclose, please do not 
%%%  include this section.

During the preparation of this work, the author(s) used [NAME OF TOOL OR SERVICE] in order to [REASON]. After using this tool or service, the author(s) reviewed and edited the content as needed and take(s) full responsibility for the content of the publication.

\newpage


\section*{\color{NavyBlue} MAIN FIGURE TITLES AND LEGENDS}

%%%  At final submission, figure files MUST be 
%%%  provided separately as high-resolution image 
%%%  files (in JPG format). All of the panels for a figure should 
%%%  be in the same file. Figures should have 
%%%  clear labels/file names (Figure 1, Figure 2, 
%%%  etc.). 

%%%  Figure titles and legends should be placed 
%%%  at the end of the main text. You do not 
%%%  need to place the figures, nor their titles 
%%%  and legends, within the main text. While 
%%%  typesetting your article, our team will 
%%%  place each figure in the best location 
%%%  based on the final layout and on your 
%%%  figure citations, e.g., (Figures 1A and 1B). 


\noindent\includegraphics[width=0.85\linewidth]{Figure1.jpg}

\subsection*{Figure 1. A brief title that describes the entire figure without citing specific panels}

The figure legend can be all one paragraph and describe the images (A), graphs (B), and plots (C), etc., together.
\newline
(A) Or each panel or group of panels can be described separately, as shown here and below.
\newline
(B) Graph of X, Y, and Z.
\newline
(C and D) If panels are grouped like this, please explicitly describe each panel, e.g., “Images showing SEM (C) and TEM (D).”
\newline
Please define all scale and error bars, and please review the Cell Press figure guidelines before submission: \href{https://www.cell.com/figureguidelines}{https://www.cell.com/figureguidelines}. Example figure created by Cassie Comeau, Cell Press.

\newpage

\section*{\color{NavyBlue} MAIN TABLES, INCLUDING TITLES AND LEGENDS}

%%%  Whenever possible, we encourage you to submit 
%%%  your main-text tables as Microsoft Word documents, 
%%%  using Word's table function. This will ensure 
%%%  the best results during conversion. Tables 
%%%  should be numbered Table 1, Table 2, etc. and 
%%%  should not include subpanels (do not use Table 1A, 
%%%  1B, etc.). Give each table a brief descriptive 
%%%  title. Table legends are optional but encouraged. 
%%%  Footnotes (superscript lowercase letters) should 
%%%  be used where necessary to indicate some feature 
%%%  of the data; please do not use bold, italic, 
%%%  colored text, or shading for this purpose. Use 
%%%  separate cells, not line breaks or spaces, for 
%%%  all discrete data elements. Small embedded 
%%%  graphics with color are OK.

%%%  Like figures, all tables must be cited within 
%%%  the main text, and our typesetting team will 
%%%  place the tables within the typeset paper at 
%%%  the appropriate locations.


\subsection*{Table 1. A table with clear organization of data}

\begin{tabular}{|l | l | l | l|} 
 \hline
 \textbf{Column 1} & \textbf{Column 2} & \textbf{Column 3} & \textbf{Column 4} \\ [1ex] 
 \hline
 Row A\textsuperscript{a} & 6 & 87,837 & 787 \\ [1ex] 
 \hline
 Row B & 7 & 78 & 5,415 \\ [1ex] 
 \hline
 Row C & 545 & 778 & 7507 \\ [1ex] 
 \hline
 Row D & 545 & 18,744 & 7,560 \\ [1ex] 
 \hline
 Row E & 88 & 788 & 6,344 \\ [1ex] 
 \hline
\end{tabular}

\bigskip

The table legend (optional) follows the table itself. The legend should be used to provide additional info that relates to the table as a whole.

\textsuperscript{a}Footnotes can be used to provide additional info on specific content within the table, such as this footnote to the first row (row A). Do not use footnotes in the table title.

\newpage

%%%  REFERENCES: We recommend placing your references in the included "references.bib" file.


\bibliography{references}



\bigskip

%%%  In your References, please include only articles 
%%%  that are published (online publication and 
%%%  preprint servers are OK). Unpublished data, 
%%%  submitted and/or accepted manuscripts, abstracts, 
%%%  and personal communications should be cited within 
%%%  the text only ("unpublished data," "data not 
%%%  shown," "Alice Smith, personal communication") 
%%%  and not included in the references list. Personal 
%%%  communication should be documented by a letter 
%%%  of permission. Whenever possible, please make 
%%%  sure your .bib file has the complete author lists 
%%%  for each item (at minimum, the first 11 authors 
%%%  listed).

\newpage




\end{document}